\section{Benchmarks}
\label{sec:benchmarks}
\begin{itemize}
  \item benchmarks - presentation of the tools: ildl plugin + numbers on FFT, miniboxing plugin extension + streams + numbers
\end{itemize}

\subsection{Methodology}

\subsection{System Setup}

\subsection{Ad hoc Data Representation Transformations}
\label{sec:benchmarks:ad-hoc}

\subsection{Retrofitting Specialization to Functions}
\label{sec:benchmarks:funcs}

The \emph{scala-streams} library introduces lazy streams as a potential
alternative to scala-views and its design is inspired by Java 8 Streams. In Java
8, streams are fundamentaly continuation-based instead of iterator-based. The
main difference between the two methodologies is that although both are lazy,
the former is lazy for producers and the latter is lazy for consumers (who can
control the iteration through the methods of the iterator). A term often used to
describe such an internal iteration scheme is ``\emph{Push
  iteration}''~\cite{obsidian,defuncpush}.  The consumer of a Push stream will
provide a consumer function of type \verb|T => Boolean| that is instantiated
into the iteration block of the stream. The loop internally is controlled by the
return value of this function and also by the existance of more elements to be
processed. The dual of a Push stream is a~\emph{Pull stream}. Every combinator
of a Pull stream will build an iterator that will propagate some effect (e.g.,
apply a function \verb|f| if this combinator is \verb|map|) to each next
element. C\#, F\# and Scala implement deferred execution over pipelines as Pull
streams. Java, on the other hand, supports Push streams by default (although it
also provides Pull capabilities through the \verb|tryAdvance| method). The
performance implications of each different design decision in stream libraries
was assessed in Biboudis et al.~\cite{biboudis_clash_2014}. The source of
inspiration for scala-streams is Nessos Streams~\cite{nessos_streams}, a library
that implements push-style streams for F\# and C\#.

\begin{lstlisting}[language=Scala, caption=Example uses of scala-streams.]
var y = Stream(xs).filter(_*2==0).toArray()
var y = Stream(xs).map(_*2).fold(0)(_+_)
var y = Stream(lines).flatMap(line => Stream(line split "\\W+")).toArray
\end{lstlisting}

The implementation of scala-streams relies on deep nesting of loop-functions in
continuation-passing style. This makes scala-streams a very good candidate to
present the performance benefits of the automated data representation
transformation of function types for pipelines that perform number crunching.

\paragraph{Description of the library. }
A Push stream type has a single constructor. The \verb|Stream| constructor takes
as a parameter a loop-function which takes as a parameter another function with
type \verb|T => bool|.

In words, a stream holds a function that represents the entire iteration over
elements of an array, with another function applied to each one.  The
\verb|bool| return value serves to indicate whether the next consumer down the
operator chain should be called on the element or not.

The iteration itself is encoded as a loop (hereafter: the \verb|iterf|
function). The \verb|Stream| class constructs such a stream with the initial
looping function passed as the \verb|iterf| function. Note the correspondence
with the Java iteration scheme. Much like the Java method \verb|Stream.of|, the
scala-streams method apply yields a consumer that contains a predefined
iteration block (\verb|iterf|) and returns a new stream of that function. Java's
\verb|tryAdvance| and \verb|forEachRemaining| (that also checks if the iteration
must proceed), correspond to the above function \verb|iterf|. The code skeleton
of \verb|ofArray| is shown below, for reference.

\begin{lstlisting}[language=scala, caption=Creation of a stream.]
@inline def apply[@miniboxed T: ClassTag](xs: Array[T]) = {
  val gen = (iterf: T => Boolean) => {
    var counter = 0
    var cont = true
    val size = xs.length
    while (counter < size && cont) {
      cont = iterf(xs(counter))
      counter += 1
    }
  }
}
new Stream(gen)
\end{lstlisting}
In Listing~\ref{lst:operators} we present two operators, an intermediate
\verb|map| and a terminal \verb|foldLeft|. \verb|map| takes as a parameter the
mapping function and returns a new stream that has updated the looping function
(the consumer of the stream) by first applying \verb|f| before passing the value
to the next transformation function.
\begin{lstlisting}[language=scala, label=lst:operators, caption=Sample operators.]
final class Stream[@miniboxed T: ClassTag](val streamf: (T => Boolean) => Unit) {

  def map[@miniboxed R: ClassTag](f: T => R): Stream[R] = 
    new Stream(iterf => streamf(value => iterf(f(value))))

  def foldLeft[@miniboxed A](a: A)(op: (A, T) => A): A = {
    var acc = a
    streamf(value => {
      acc = op(acc, value)
      true
    })
    acc
  }
}
\end{lstlisting}
It is of high importance to provide an implementation that works well under both
reference types and primitives.  \aggelos{Trying to use map as the base example
  of the transformation. Will try again tomorrow to dissasemble
  https://gist.github.com/biboudis/00b22abea986e496f060 obtained by
  "-Xprint:minibox-commit"}
\paragraph{Assessment. } \aggelos{TODO}.
