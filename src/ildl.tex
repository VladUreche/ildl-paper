\section{Incremental Late Data Layout}
\label{sec:ildl}

This section will present the Incremental Late Data Layout (ILDL) transformation, by extending the global, all-or-nothing Late Data Layout (LDL) mechanism with the necessary extensions for ad-hoc programmer-driven transformations.

% \subsection{Data Representation Transformations}

% DRT - when the data has more than one representation - generally, the representation can be either more flexible or more efficient
Data can usually be represented in several ways, some more efficient and others more flexible. For example, the integers can use either the primitive (unboxed) value encoding, which is more efficient, or the object-based (boxed) encoding, which is more flexible, allowing the value to be assigned to supertypes, to handle method calls and to instantiate erased generics. However, the extra flexibility comes at a price: boxed integers are allocated on the heap so they need to be garbage-collected later. Also, any operations on boxed integers are performed via indirect access to the values. There is a tension between the two representations.

% From the language perspective: Expose this difference vs transform in the compiler
From a language perspective, there are two approaches to exposing multiple representations: either present each representation as different and incompatible type in the language, as Java and C do, or hide the difference and present a single high-level type, as ML, Haskell and Scala do. Although the second approach is more popular among programmers for its simplicity, it puts more burden on the compiler.

% DRTs- two steps: decide on the representation of each value and introduce coercions
% Choosing the representation -- simple, but mind the generics -- for our case, we need to select the values
When translating source code written against high-level types, a compiler needs to choose the data representation of each value and to introduce coercions that transform one representation to another when necessary. For example, since only boxed integers can instantiate generics, any unboxed integer going into a generic container, such as a list of integers, needs to be converted to the boxed representation. This work is done in the compiler pipeline, in so-called data representation transformations.

% Introducing coercions -- optimality question
Furthermore, what makes things even more complicated is the fact that coercions need to be introduced optimally, so the program performance does not suffer. For example, boxing a value can be done in either one step, |box(...)|, or in three steps, sub-optimally: |box(unbox(box(...)))|. Of course, such an obvious sub-optimality is easy to correct, but there are more subtle cases, where devising a general optimization rule is very difficult. This motivated the work on the Late Data Layout mechanism.

\subsection {Late Data Layout}

% In this context, Late data laout was chosen for its ability to selectivity pick values to be transformed and the optimality in introducing coercions
Late Data Layout (LDL) is a general mechanism for data representation transformations. It allows selectively picking the representation for each value and automatically introduces the necessary coercions in a consistent and optimal way. Selectively choosing the representation of each value can be done either by the user, through annotations or by the compiler, based on predefined rules.

The LDL mechanism is the underlying transformation used to unbox primitive types in Scala, to implement value class inlining and to specialize classes using the miniboxed encoding. This makes LDL a flexible and reliable mechanism for translating generics, and thus a natural choice for our starting point in ad-hoc data representation transformations.

% What does it do? Starting from a high-level type (or concept), such as |scala.Int|, it transforms all references to low-level representations
Using LDL, a language can expose high-level types (called high-level concepts in the LDL terminology), such as the integer type exposed by Scala, |Int|, which can represent either a boxed or unboxed value in the low-level bytecode. In the following example, we have values of types |Int| and |Any|, where |Any| is the top of the Scala type system, also being a supertype of |Int|:

\begin{lstlisting-nobreak}
val i: Int = 1
vla j: Int = i
val k: Any = j
\end{lstlisting-nobreak}

During compilation, the LDL mechanism transforms the program such that all values use either the unboxed |int| or boxed |java.lang.Integer| representation, and the necessary coercions between these two representations have been introduced consistently and optimally. For example, the code above is translated to:

\begin{lstlisting-nobreak}
val i: `int` = 1
vla j: `int` = i
val k: Any = Integer.valueOf(j)
\end{lstlisting-nobreak}

%How does LDL work? -- injects representation information in the types and uses them to introduce coercions
\subsubsection{The LDL Transformation Phases.}

%3 phases: Inject, Coerce, Commit.
The LDL mechanism transforms the data representation in three phases, with each phase being responsible for a characteristic of the transformation. The three phases are \inject{}, \coerce{} and \commit{}. In our examples, we show the equivalent source code for the program abstract syntax trees (ASTs) after each of these phases.

% Inject
The \inject{} phase is responsible for marking each value with its desired representation. In the case of primitive integer unboxing, the annotation is |@unboxed|, and it signals a value should be stored in the unboxed |int| representation. As an optimization, instead of adding another |@boxed| annotation, values that are not marked are automatically considered as boxed. Following the \inject{} phase, the previous example will be transformed to:

\begin{lstlisting-nobreak}
val i: `@unboxed` Int = 1 // Int can be unboxed => add @unboxed annotation
vla j: `@unboxed` Int = i // Int can be unboxed => add @unboxed annotation
val k: Any = j                  // Any cannot be unboxed => requires boxed value
\end{lstlisting-nobreak}

The \inject{} phase gives LDL the selective nature, allowing it to mark each individual value with its representation. For example, it would have been equally correct if the marking rules decided that |j| should be boxed, in which case its type would not have been marked. One of the tricks employed by the LDL transformation is that, during the \inject{} phase boxed and unboxed values are still compatible, so there is no need for coercions.

% Coerce
The \coerce{} phase, as its name suggests, introduces coercions. This is done by changing the annotation semantics: annotated types become incompatible among themselves and with their un-annotated types. This change in the annotation semantics corresponds to introducing the different representations: each annotation corresponds to a representation, and representations are not compatible with each other. With this change, an assignment from one representation to another will have mismatching types. Therefore, by re-type-checking the tree, the \coerce{} phase can detect representation mismatches and can patch them using coercions. In the example, the last line contains such a mismatch:

\begin{lstlisting-nobreak}
val i: @unboxed Int = 1 // expected: @unboxed, found: @unboxed
vla j: @unboxed Int = i // expected: @unboxed, found: @unboxed
val k: Any = `box`(j)             // expected:  <@boxed>, found: @unboxed => box value
\end{lstlisting-nobreak}

The \coerce{} phase brings the optimal nature of the LDL transformation. It does this by introducing coercions only when a representation mismatch occurs and by delaying coercions as much as possible. This is done thanks to an intra-procedural data flow analysis done in the type system: top-down type propagation from symbol definitions to their uses and bottom-up propagation of expected types, part of local type inference. We will not go into the details of how optimality is achieved, but we point our readers to a paper presenting LDL \cite{ldl} for further explanations. Still, for the purpose of ad-hoc data representation transformations, it is important to rely on a mechanism that does not introduce redundant coercions between the different representations.

% Commit
The \commit{} phase is responsible for introducing the actual representations. In the case of primitive unboxing, |@unboxed Int| is replaced by |int| and |Int|, which is considered boxed, is replaced by |java.lang.Integer|. The |box| and |unbox| coercions are also replaced by the creation of objects and, respectively, by the extraction of the unboxed value.

\begin{lstlisting-nobreak}
val i: `int` = 1
vla j: `int` = i
val k: Any = `Integer.valueOf`(j)
\end{lstlisting-nobreak}

The \commit{} phase is responsible for the consistency of the transformation. Since the program abstract syntax tree (AST) has been checked by the type-system extended with representation semantics, the \commit{} phase is guaranteed to correctly handle the value representations and to correctly coerce between them. This allows the \commit{} phase to be a very simple, syntax-based, transformation over the program abstract syntax tree (AST).

% Object-oriented support
\subsubsection{Object-Oriented Patterns.}

Aside from introducing coercions, data representation transformations must handle object-oriented patterns, such as method calls and subtyping. Not all representations can be used with these patterns, such as, for example, calling the |toString| method on the unboxed |int| representation:

\begin{lstlisting-nobreak}
val a: `@unboxed Int` = 1
println(a.toString)
\end{lstlisting-nobreak}

Fortunately, LDL allows the transformations to handle these cases with ease: by assigning the non-annotated type to the most flexible representation and requiring the method call receivers to have non-annotated types, LDL makes it easy to correctly handle object-oriented patterns. In the case of calling the |toString| method, the |@unboxed Int| receiver will be boxed, so it can act as the receiver of the method call. Subtyping is handled identically, by requiring the most flexible representation:

\begin{lstlisting-nobreak}
val a: @unboxed Int = 1
println(`box`(a).toString)
\end{lstlisting-nobreak}

To improve performance, the LDL mechanism also supports extension methods. For example, if the |extension_toString| method is available for the unboxed |int| representation, there is no need to convert it to the boxed representation just to call |toString| on it. Instead, the call takes place without a representation conversion:

\begin{lstlisting-nobreak}
val a: @unboxed Int = 1
println(`extension_toString`(a))
\end{lstlisting-nobreak}

% Limited code scopes
\subsubsection{Support for Generics.}
The Late Data Layout mechanism is agnostic to generics. This means that, depending on the implementation of generics, the mechanism can either propagate the annotations for type arguments or not. For example, if generics are erased, a list of integers will have type |List[Int]|, since values need to be boxed. If generics are unboxed and reified, the list type will be |List[@unboxed Int]|. In the original paper \cite{ldl}, the authors show examples of both cases, where annotations are propagated inside generics and when they are not, showing how the LDL mechanism adapts seamlessly to either case.

Having seen the Late Data Layout mechanism at work for unboxing primitive types, we can now look at how it can be extended to handle ad-hoc programmer-driven data representation transformations.

\subsection{Incremental Late Data Layout}

Global vs local/ad-hoc -- relationship between LDL and iLDL + separate compilation

Where does the transformation take place -- difference between high-level signatures and low-level signatures \ldots

Injection -- before storing signatures, bridge, coerce and commit come later.

%%%%%%%%%%%%%%%%%%%%%%%%%%%%%%%%%%%%%%%%%%%%%%%%%%%%%%%%%%%%%%%%%%%%%%%%%%%%%
\subsection{Persisting the Data Representation information}
\label{sec:ildl:signatures}

% An important question in DRTs: how to modify signatures
%  -- keeping track of transformed methods (do I have a transformed version?)
High-level signatures keep the high-level type, annotated with the transformation description object.

% Persisting the transformation decisions = persisting annotations
%  -- Usage across compilations
%  -- Tracking the transformer
%  -- interoperating between ``islands'' of modified code
%      -- using the same encoding
%      -- using different encodings
We need this in order to save the decisions made during injection.

\subsubsection{Tracking the Transformed Values}
This allows inserting coercions.

\subsubsection{Two Types, One Representation}

What if we have two types that erase to the same representation? Could a method where the first type, say (Int, Int) was converted to long be called by code that was encoding |(Float, Float)| as a long integer, thus compromising the semantics of the language?

The answer is no, since the type system protects us from doing this. The annotations, before the coerce phase, only bear the final representation, but the type exposed is the high-level type, not the representation. Therefore, the error message would make it very clear that the code attempted to call a method accepting a pair of integers but passed a pair of floating-point numbers.

\subsubsection{One Type, One Representation, Compatibility}

Another question we may ask ourselves is what we have the |ildl| marker method called from two different locations with the same transformation description object? Would they need to convert to the high-level (and inefficient type)? The answer is no, since the method signature will have the high-level type annotated with the representation, which will match the arguments for the call.

\subsubsection{One Type, Two Representations}

Finally, a third question one may ask is what if we had a single type, such as |GaussianInteger|, which could be represented as either |(Int, Int)| or |Long|, and attempted to call between parts of the code using the two representations. The answer is that, keeping the high-level signature would allow the ildl transformation to know that it needs to convert from one representation to another: it would first decode the first representation to expose the high-level type, which it would then re-encode with the second representation.

%%%%%%%%%%%%%%%%%%%%%%%%%%%%%%%%%%%%%%%%%%%%%%%%%%%%%%%%%%%%%%%%%%%%%%%%%%%%%
\subsection{Handling Generics and The Object Model}
\label{sec:ildl:generics}

% 2 Aspects \ldots
\subsubsection{Generics}

% Generics
%  -- Nesting datastructures
Can we transform |List[(Int, Int)]| into |List[Long]|. In the general case, no, for two reasons:
\begin{itemize}
  \item since it could break aliasing (for mutable data structures)
  \item since there is no method available to transform from one representation to the other
\end{itemize}

Solution: we could either add methods for transforming data structures in the transformation description object or, if there is no such method available, the correctness-preserving approach is not to transform the representation inside generics.


\subsubsection{The Object Model: Subtyping and Overriding}
% Overriding - introduce an additional phase in LDL: bridge
We add bridges, therefore the ILDL transformation has four phases instead of three: inject, bridge, coerce and commit.


%%%%%%%%%%%%%%%%%%%%%%%%%%%%%%%%%%%%%%%%%%%%%%%%%%%%%%%%%%%%%%%%%%%%%%%%%%%%%
\subsection{Preserving the Semantics}
\label{sec:ildl:semantics}

% Semantics changes
\subsubsection{Methods and Operators of the Object}
% Operators
%  -- transformation tracker
We can always decode the representation to expose the high-level type and make the call based on that. Yet, this is inefficient. The solution is to short-circuit the calls to the operators and perform them directly on the encoded value, much like extension methods work for value classes.

Short-cutting methods are located in the transformation description object, which needs to contain:
\begin{itemize}
  \item transformations between the high-level type and the representation (and back);
  \item (optional) transformations between generic containers with the high-level type (|List[(Int, Int)]|) and the generic container with the improved representation (|List[Long]|);
  \item (optional) operators and methods on the encoded representation, such as |+|, |-|, |*|, |/|, |%| and |norm|.
\end{itemize}

Operators need not required their arguments to use the high-level type: for example, when defining the |+| operator, we need to force the second operator to use the high-level type |(Int, Int)|. Instead, we accept a value of type |@encoded Long|, which allows it to accept the optimized representation for the other value as well. Interestingly, this does not prevent it from accepting the high-level type |(Int, Int)|: based on annotations, the ildl transformation knows the |(Int, Int)| value must be converted to a |Long| value.

\subsubsection{Semantics of the Transformed Type}
%  -- Liskov substitution principle
reviewers may argue that data containers are generic and flexible in order to ease evolution of the data structures. Therefore transforming non-final containers might lose their semantics, since we fix the semantics of operators, which otherwise would have been dynamically dispatched to the most specific implementation. But that can be argued against using the Liskov principle -- if a subclass of the container modifies the behavior of an operator, even the original program's semantics may now be incorrect, therefore our transformation won't make correctness worse. Therefore, according to the Liskov principle, as long as the updated (static) operators are semantically equivalent to the data container's, the transformation will not affect the program semantics.

