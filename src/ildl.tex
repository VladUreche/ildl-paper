\section{Incremental Late Data Layout}
\label{sec:ildl}
\begin{itemize}
  \item theory of ldl - data representation transformation - coercions, optimality
  \item injection + signatures, persistence
\end{itemize}

% Executive summary + vision

\subsection{Data Representation Transformations}

% DRT - when the data has more than one representation - generally, the representation can be either more flexible or more efficient

% From the language perspective: Expose this difference vs transform in the compiler

% DRTs- two steps: decide on the representation of each value and introduce coercions

% Choosing the representation -- simple, but mind the generics!
%  -- for our case, we need to select the values

%% TODO: Find compelling example :)

% Introducing coercions -- optimality question

\subsection {Late Data Layout}

% In this context, Late data laout was chosen for
%  - selectivity
%  - optimality
%  - availability (in Scala)

% What does it do? Starting from a high-level concept, such as scala.Int, it transforms all references to low-level reprs

% How does LDL work? -- injects representation information in the types and uses them to introduce coercions

% 3 phases: Inject, Coerce, Commit.

% Inject

% Coerce

% Commit

\subsection{Incremental Late Data Layout}

% Global vs local/ad-hoc -- relationship between LDL and iLDL

% An important question in DRTs: how to modify signatures
%  -- keeping track of transformed methods (do I have a transformed version?)
%  -- overriding
%  -- clashes

% Persisting the transformation decisions = persisting annotations

% Overriding - introduce an additional phase in LDL: bridge

% Clashes - reported during erasure