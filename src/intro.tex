\section{Introduction}
\label{sec:intro}

% Objects = encapsulated data + API. Extension methods = ad hoc additions of the API
An object encapsulates code and data and exposes an interface. Modern
language facilities, such as extension methods, type classes and
implicit conversions allow programmers to evolve the object interface
in an ad hoc way, by adding new methods and operators. For example, in
Scala, we can use an implicit conversion to add the multiplication
operator to pairs of integers, with the semantics of complex number
multiplication:

\begin{lstlisting-nobreak}
scala> (0, 1) * (0, 1)
res0: (Int, Int) = (-1, 0)
\end{lstlisting-nobreak}

% Interface okay, what about the object layout?
Unlike evolving the interface, no general mechanism in modern
languages is capable of evolving an object's encapsulated data. The
data representation is assumed fixed so the compiled code contains hard
references to the encapsulated data, encoded according to a convention
known as the \emph{object layout}. For example, methods encapsulated
by the generic pair class, such as |swap| and |toString|, rely on the
existence of two generic fields, erased to |Object|. This leads to
an inefficient storage in our case, as the integers need to be boxed, producing as many as
3 heap objects for each ``complex number'': the two boxed integers and
the pair container. What if, for a part of  our program, instead of the pair, we
concatenated the two 32-bit integers into a 64-bit long integer, that
would represent the ``complex number''? We could pass it by value,
completely sidestepping the need to allocate memory and to collect it
later. Additionally, what if we could also add
functionality, such as arithmetic operations, to our ad hoc complex
numbers, all without any overhead? Finally, what parts of this
transformation could be automated? %and what parts need domain-specific
%programmer intervention?

% Object layout transformations in dynamic language vms => slow (due to profiling and recompilation)
Object layout transformations are common in dynamic language virtual
machines, such as V8 and Truffle. These virtual machines profile
values at run-time and make optimistic assumptions about the shape of
objects. This allows them to automatically optimize the object layout
in the heap, at the cost of recompiling of all the code that references
 the old object layout.
%However, in practice, the effort pays off.
If later in the execution, the assumptions prove too optimistic, the
virtual machine needs to revert to the more general (and less
efficient) object layout, again recompiling all the code that contains
hard references to the optimized layout. As expected, this comes with
important overheads. Thus, runtime system decisions to change the
low-level object layout are both global, affecting all code that
assumes a certain layout, and expensive, due to recompilation.

%%\yannis{I don't think it's accurate to put the slowness down to that.}
%This makes dynamic languages several times slower than
%compiled, statically typed languages.

% Object layout transformations in statically typed languages => primitive unboxing, value classes and specialization
Since transforming the object layout at run-time is expensive, a
natural question to ask is whether we can leverage the
statically-typed nature of a programming language to optimize the
object layout? The answer is yes. Transformations such as ``class
specialization'' and ``value class inlining'' transform the object
layout in order to avoid the creation of heap objects. However, both
of these transformations take a global approach: when a class is
marked as specialized or as a value class (and assuming it satisfies the
semantic restrictions) it is transformed at its definition site. Later
on, this allows all references to that class, even in separately
compiled sources, to be optimized by the compiler. On the other hand,
if a class is not marked at its definition site, retrofitting
specialization or the value class status is impossible, as it would
break many non-orthogonal language features, such as dynamic dispatch,
inheritance and generics.

% Object layout transformations in statically typed languages => not ad hoc
Therefore, although transformations in statically typed languages can
optimize the object layout, they do not meet the ad hoc criterion:
they cannot be retrofitted later and have a global, all-or-nothing
nature. For instance, in Scala, the generic pair class is specialized
but not marked as a value class. As a result, the representation is
not fully optimized, still requiring a heap object per pair. Even
worse, specialization and value class inlining are mutually exclusive,
making it impossible to optimally represent our ``complex numbers''
as values even if we had complete control over the Scala library.

% secret ingredient => the domain-specific information provided by the programmer
Furthermore, a change of data representation may only be applicable in
specific client code. In our ``complex numbers'' abstraction, we only
use a fraction of the flexibility provided by the library tuples, and
yet we have to give up all the code optimality. Even worse, for our
limited domain and scope, we are aware of a better representation, but
the only solution is to transform the code by hand, essentially having
to choose between an obfuscated or a slow version of the code. What is
missing is a largely automated and safe transformation that allows us
to use domain-specific knowledge in order to mark a scope where the
``complex numbers'' use an alternative object layout, effectively
specializing that part of our program.

% Coincidentally, this is what we're proposing in this paper...
In this paper we present such an automated transformation that allows
programmers to safely change the data representation in limited,
well-defined scopes that can include anything from expressions to
method, class and package definitions, while maintaining strong
correctness guarantees in terms of non-orthogonal language features,
such as dynamic dispatch, inheritance and generics across separate
compilations.
%% \yannis{I don't think any reader will appreciate this at this point.}
% Interestingly, our method does not rely on either
%specialization or the value class transformation. Instead, it extends
%the Late Data Layout transformation, which underpins and generalizes
%specialization and value classes, in order to allow ad hoc changes to
%the object representation.
To gain the most benefit, our transformation uses the programmers'
intimate knowledge of the transformed scope, allowing them to specify
the exact alternative representation and the static operations it
should expose, while completely automating all the tedium involved in
safely transforming the code.

This way, the programmer is responsible for correctly stating (a)
what is the data representation transformation and (b) the program
scope to which it is applicable. Our mechanism is then
responsible for (1) automatically deciding when to apply the
transformation and when to revert it, in order to ensure correct
interchange between representations, (2) enriching the transformation
with automatically computed bridge code that ensures correctness
relative to overriding and dynamic dispatch and (3) persisting
the necessary metadata to allow transformed program scopes in
different source files and compilation runs to communicate using
the optimized representation, a property we call composition.
Thus, our approach adheres to the principle of separating the
reusable, general and provably correct mechanism from the
programmer-driven policy, which may contain incorrect
decisions \cite{lampson-mechanism-policy}.
%(\S\ref{sec:ildl:discussion}).

Our main contributions are:
\begin{compactitem}
  \item Defining the ad hoc data representation problem, which, to the
    best of our knowledge, has not been addressed at all in the
    literature (\S\ref{sec:problem});
  \item Presenting the extensions that allow global data
    representation transformations (\S\ref{sec:drt}) to be used as
    ad hoc programmer-driven transformations (\S\ref{sec:ildl});
  \item Benchmarking two Scala compiler plugins based on our solution
    and showing they can speed up user programs by factors between 2
    to 14x (\S\ref{sec:benchmarks}).
\end{compactitem}

%% \yannis{I recommend leaving this text out. I think it doesn't add
%% much. I integrated one important point (lampson) earlier.}
%The result of our work is an intuitive interface over an optimal,
%consistent and composable programmer-driven data representation
%transformation, where the composition works not only across source
%files but also across separate compilation
%(\S\ref{sec:ildl:signatures}). Furthermore, the transformation adheres
%to the principle of separating the reusable, general and provably
%correct mechanism from the programmer-driven policy, which may contain
%incorrect decisions \cite{lampson-mechanism-policy}
%(\S\ref{sec:ildl:discussion}). Overall, we feel the value brought by
%the transformation surpasses the sum of its individual components,
%opening new directions in compiler customization and programmer-driven
%transformation.

% The following section will describe the problem of ad hoc data representation transformations.
