\section{Introduction}
\label{sec:intro}

% Objects = encapsulated data + API. Extension methods = ad-hoc additions of the API
An object encapsulates code and data and exposes an interface. Modern
language facilities, such as extension methods, type classes and
implicit conversions allow programmers to evolve the object interface
in an ad-hoc way, by adding new code. For example, in Scala, we can
use an implicit conversion to add the multiplication operator to pairs
of integers, with the semantics of complex number multiplication:

\begin{lstlisting-nobreak}
scala> (0, 1) * (0, 1)
res0: (Int, Int) = (-1, 0)
\end{lstlisting-nobreak}

% Interface okay, what about the object layout?
Unlike evolving the interface, no general mechanism in modern
languages is capable of evolving an object's encapsulated data. The
data representation is assumed fixed. Compiled code contains hard
references to the encapsulated data, encoded according to a convention
known as the \emph{object layout}. For example, methods encapsulated
by the generic pair class, such as |swap| and |toString|, rely on the
existence of two generic fields, erased to |Object|. This leads to
inefficiencies, as the integers need to be boxed, producing as many as
3 heap objects for each ``complex number'': the two boxed integers and
the pair container. What if, for our program, instead of the pair, we
concatenated the two 32-bit integers into a 64-bit long integer, that
would represent the ``complex number''? We could pass it by value,
completely sidestepping the need to allocate memory and to collect it
later. Additionally, what if we could also add the desired
functionality, such as arithmetic operations, to our ad-hoc complex
numbers, all without any overhead?  Finally, what if this
transformation would be completely automated?

% Object layout transformations in dynamic language vms => slow (due to profiling and recompilation)
Object layout transformations are common in dynamic language virtual machines, such as V8 and Truffle. These virtual machines profile values at run-time and make optimistic assumptions about the shape of objects. This allows them to automatically optimize the object layout, but forces the recompilation of all the code that references the old object layout. However, in practice, the effort pays off. Still, if later in the execution, the assumptions prove too optimistic, the virtual machine needs to revert to the more general (and less efficient) object layout, again recompiling all the code that contains hard references to the optimized object layout. As expected, this comes with important overheads, both for profiling and for the recompilation. This makes dynamic languages several times slower than compiled, statically typed languages.

% Object layout transformations in statically typed languages => primitive unboxing, value classes and specialization
Since transforming the object layout at run-time is expensive, a natural question to ask is whether we can leverage the statically-typed nature of a programming language to optimize the object layout? The answer is yes. Transformations such as class specialization and value class inlining transform the object layout in order to avoid the creation of heap objects. However, both of these transformations take a global approach: when a class is marked as specialized or as a value class, it needs to conform to certain restrictions and is thoroughly transformed at its definition site. Later on, this allows all references to that class, even in separately compiled sources, to be optimized by the compiler. On the other hand, if a class is not marked at its definition site, retrofitting specialization or the value class status is impossible, as it would break many non-orthogonal language features, such as dynamic dispatch, inheritance and generics.

% Object layout transformations in statically typed languages => not ad-hoc
Therefore, although transformations in statically typed languages can optimize the object layout, they do not meet the ad-hoc criterion: they cannot be retrofitted later and have a global, all-or-nothing nature. In Scala, the pair is specialized but not marked as a value class, so the representation is not fully optimized, still requiring a heap object per pair. Even worse, specialization and value class inlining are mutually exclusive, making it impossible to represent our ``complex numbers'' as a long integer value even if we had complete control over the Scala library.

% secret ingredient => the domain-specific information provided by the programmer
In our ``complex numbers'' abstraction, we only use a fraction of the flexibility provided by the library tuples, and yet we have to give up all the code optimality. What makes this even worse is that, for our limited domain and scope, we are aware of a better representation, but the only solution is to transform the code by hand, essentially having to choose between an obfuscated or a slow version of the code. What if there was an automated and safe transformation that allowed us to use our domain-specific knowledge in order to mark a scope where the ``complex numbers'' use an alternative object layout, effectively specializing that part of our program?

% Coincidentally, this is what we're proposing in this paper...
In this paper we present an automated transformation that allows programmers to safely change the data representation in limited, well-defined scopes that can include anything from an expression to method, class and package definitions, while maintaining the strong correctness guarantees in terms of non-orthogonal language features, such as dynamic dispatch, inheritance and generics across separate compilations. Interestingly, our method does not rely on either specialization or the value class transformation. Instead, it extends the Late Data Layout transformation, which underpins and generalizes specialization and value classes, in order to allow ad-hoc changes to the object representation. To gain the most benefit, our transformation uses the programmers' intimate knowledge of the transformed scope, allowing them to specify the exact alternative representation and the static operations it should expose, while completely automating all the tedium involved in safely transforming the code.

Our main contributions are:
\begin{itemize}
  \item Defining the ad-hoc data representation problem, which, to the best of our knowledge, has not been addressed at all in the literature (\S\ref{sec:problem});
  \item Presenting the extensions that allow global data representation transformations (\S\ref{sec:drt}) to become ad-hoc programmer-driven transformations (\S\ref{sec:ildl}) and suggesting how these extensions can be exposed to the programmer (\S\ref{sec:impl});
  \item Benchmarking two Scala compiler plugins based on our solution and showing they can speed up user programs by factors of 2 to 14x (\S\ref{sec:benchmarks}).
\end{itemize}

\vlad{TODO: Talk about composability and safety!!!}

The following section will describe the problem of ad-hoc data representation transformations.
